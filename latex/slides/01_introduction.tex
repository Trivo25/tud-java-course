\input{course_definitions}	% nothing to do here
\usepackage[german]{babel}

\usepackage[utf8]{inputenc}



\lstset{
  language = Java,
  showspaces = false,
  showtabs = false,
  showstringspaces = false,
  escapechar = @,
  belowskip=-1.5em
}

\def\ContinueLineNumber{\lstset{firstnumber=last}}
\def\StartLineAt#1{\lstset{firstnumber=#1}}
\let\numberLineAt\StartLineAt

\makeatletter
 % TODO modify this if you have not already done so


% meta-information
\usepackage{tikz}
\usepackage{hyperref}
\hypersetup{
	colorlinks=true,
	linkcolor=darkgray,
	urlcolor=blue,
}


\title{Java}
\subtitle{Object Oriented  Programming}
\date{\today}
\author{Florian Kluge, Moritz Schulz}
\institute{Florian.Kluge@mailbox.tu-dresden.de, Moritz.Schulz2@mailbox.tu-dresden.de}
% \titlegraphic{\hfill\includegraphics[height=1.5cm]{logo.pdf}}

\begin{document}
\maketitle


\begin{frame}{Overview}
	\setbeamertemplate{section in toc}[sections numbered]
	\tableofcontents[hideallsubsections]
\end{frame}


% ################# ORGANISATION #################


\section{Organisation}

\subsection{Who are we?}
\begin{frame}{Who are we?}

	\begin{columns}
		\column{0.50\textwidth}
		Florian Kluge\\
		Florian.Kluge@mailbox.tu-dresden.de



		\column{0.50\textwidth}
		Moritz Schulz\\
		Moritz.Schulz2@mailbox.tu-dresden.de\\
		@schokotets auf Telegram

	\end{columns}

\end{frame}

\subsection{What are we doing here?}
\begin{frame}{What are we doing here?}
	What are we doing here?
	\begin{itemize}
		\item Introduction to programming
		\item Getting to know the basics of Java
		\item Preparation for upcoming courses (e.g 'Softwaretechnologie', 2nd Semester)
	\end{itemize}
\end{frame}

\subsection{Structure}
\begin{frame}{Structure}
	\begin{itemize}[<+->]
		\item ~14 lessons
		\item Thursday, 13:00 - 14:30
		\item APB/006 (right here)
		\item 3G-Rule
		\item Attendance list
	\end{itemize}
\end{frame}

\begin{frame}{Structure}
	\begin{itemize}[<+->]
		\item This course is *voluntarily*
		\item Bored and want to leave? No problem!
		\item[]	.. but please contact us so we can invite students from the waiting list
		\item If you don't attend the course for two weeks in a row without notice we will give your slot to other students
	\end{itemize}
\end{frame}

% ################# Why Java? #################

\section{Why Java?}
\begin{frame}{Why Java?}
	\begin{itemize}
		\item Widely used programming language
		\item Introduction to object oriented programming (OOP)
		\item Platform-independent
		\item ... and much more
	\end{itemize}
\end{frame}

\subsection{Use cases}
\begin{frame}{Use cases}
	\begin{itemize}
		\item Android development
		\item Web applications
		\item Desktop GUI applications
		\item ... and much more
	\end{itemize}
\end{frame}


\subsection{Experience}
\begin{frame}{Do you have any programming experience already?}
	\begin{center}
		Do you have any programming experience already?\\
		\url{https://trivo25.github.io/tud-java-course/poll.html}\\
		or\\
		\url{https://strawpoll.com/6uh45fcvx}
	\end{center}
\end{frame}


% ################# Time to get started! ..almost #################

\section{Time to get started! ..almost}

\begin{frame}[fragile]{Time to get started! ..almost}
	\begin{center}
		Java OpenJDK 11 \url{https://adoptium.net/}
		\\
		Did you install it correctly? Time to find out!
	\end{center}

	\begin{lstlisting}[language=bash]
		$ javac --version
		> javac 11.0.12
	\end{lstlisting}
\end{frame}



\begin{frame}[fragile]{Time to get started! ..almost}
	\begin{center}
		Doesn't work? :(
		Use an online compiler!

		\url{https://www.jdoodle.com/online-java-compiler/}
	\end{center}
\end{frame}



% ################# lets go #################

\section{Let's go!}

\subsection{Your first task}
\begin{frame}[fragile]{Your first task}
	\begin{itemize}[<+->]
		\item Create a new folder
		\item Open the terminal and navigate into that folder using \lstinline[columns=fixed]{$ cd /to/my/folder}
		\item Create a new file by either typing\\ \lstinline[columns=fixed]{$ touch helloWorld.java}\\
		      or right-clicking in your folder \lstinline[columns=fixed]{Right click -> new -> text document}\\
		      and save it as a \lstinline[columns=fixed]{.java} file
	\end{itemize}
\end{frame}

\subsection{Your first task}
\begin{frame}[fragile]{Your first task}
	\begin{itemize}[<+->]
		\item now its time to write your first piece of code!
		\item[] \lstinputlisting[language=java]{../code_samples/HelloWorld.java}
	\end{itemize}
\end{frame}

\subsection{How to execute a java program}
\begin{frame}[fragile]{How to execute a java program}
	what we have to do now..
	\begin{itemize}[<+->]
		\item telling \lstinline[columns=fixed]{javac} to compile our \lstinline[columns=fixed]{helloWorld.java} file into a \lstinline[columns=fixed]{helloWorld.class}
		\item \lstinline[columns=fixed]{.class} files are 'bytecode' for the Java Virtual Machine (JVM)
		\item with \lstinline[columns=fixed]{$ java helloWorld} we can finally execute our first program!
	\end{itemize}
\end{frame}

\subsection{How to execute a java program}
\begin{frame}[fragile]{How to execute a java program}
	\begin{lstlisting}[language=bash]
		$ java helloWorld
		> Hello World!
	\end{lstlisting}
\end{frame}


\subsection{Time to play around}
\begin{frame}[fragile]{Time to play around}
	your next task
	\begin{itemize}[<+->]
		\item change the text you want to print in the \lstinline[columns=fixed]{helloWorld.java} file
		\item re-compile it into a \lstinline[columns=fixed]{.class} file and execute it again!
	\end{itemize}
\end{frame}


\subsection{What are we actually doing?}
\begin{frame}[fragile]{What are we actually doing?}
	\begin{itemize}[<+->]
		\item we are telling the computer what do to
		\item we list instructions for the computer
	\end{itemize}
\end{frame}


\subsection{Task numero 2!}
\begin{frame}[fragile]{Task numero 2!}
	\begin{center}
		Let's add a variable of type \lstinline[language=Java]{String}!
	\end{center}
\end{frame}

\subsection{Task numero 2!}
\begin{frame}[fragile]{Task numero 2!}
	\begin{center}
		\lstinputlisting[language=java]{../code_samples/VariableString.java}
	\end{center}
\end{frame}


\subsection{Task numero 2!}
\begin{frame}[fragile]{Task numero 2!}
	\begin{itemize}[<+->]
		\item We can re-use variables
		\item We can store data in them
	\end{itemize}
\end{frame}


\subsection{Task numero 3!}
\begin{frame}[fragile]{Task numero 3!}
	\begin{center}
		Let's talk to the console and read our input!
	\end{center}
\end{frame}



\subsection{Task numero 3!}
\begin{frame}[fragile]{Task numero 3!}
	\begin{center}
		\lstinputlisting[language=java]{../code_samples/VariableStringName.java}
	\end{center}
\end{frame}


\subsection{Task numero 4!}
\begin{frame}[fragile]{Task numero 3!}
	\begin{itemize}[<+->]
		\item Besides \lstinline[language=Java]{String}s we also have variables of type \lstinline[language=Java]{int}\\
		\item \lstinline[language=Java]{int} represent whole numbers, like \lstinline[language=Java]{1, 18, 1337 or 420360}\\
		\item We can calculate \lstinline[language=Java]{int} with operators like \lstinline[language=Java]{+}, \lstinline[language=Java]{-}, \lstinline[language=Java]{*} and many more
	\end{itemize}
\end{frame}


\subsection{Task numero 4!}
\begin{frame}[fragile]{Task numero 3!}
	We now can..
	\begin{itemize}[<+->]
		\item Display text in the console \lstinline[language=Java]{System.out.println("Hello word!");}
		\item Declare variables like \lstinline[language=Java]{int} or \lstinline[language=Java]{String}
		\item Read input from the conole
		\item .. and know operators like \lstinline[language=Java]{+, - or *}
		\item[] Okay, what now?
		\item[] Let's build a calculator!
	\end{itemize}
\end{frame}
% ################# ending first lesson #################

\section{That's it (at least for today)}

\subsection{}
\begin{frame}{What will we do next lesson?}
	\begin{itemize}
		\item Deep dive into (more) variables and their operators
		\item Introducing functions and control flow
		\item and build more cool things!
	\end{itemize}
\end{frame}

\subsection{}
\begin{frame}{Links and resources}
	\begin{center}
		\url{https://trivo25.github.io/tud-java-course/}
	\end{center}

	\begin{figure}[htbp]
		\centerline{\includegraphics{assets/qr.png}}
		\label{fig}
	\end{figure}
\end{frame}

% nothing to do from here on
\end{document}
